

% This document was generated by the publish-function
% from GNU Octave 4.2.1



\documentclass[10pt]{article}
\usepackage{listings}
\usepackage{mathtools}
\usepackage{amssymb}
\usepackage{graphicx}
\usepackage{hyperref}
\usepackage{xcolor}
\usepackage{titlesec}
\usepackage[utf8]{inputenc}
\usepackage[T1]{fontenc}
\usepackage{lmodern}


\lstset{
language=Octave,
numbers=none,
frame=single,
tabsize=2,
showstringspaces=false,
breaklines=true}


\titleformat*{\section}{\Huge\bfseries}
\titleformat*{\subsection}{\large\bfseries}
\renewcommand{\contentsname}{\Large\bfseries Contents}
\setlength{\parindent}{0pt}

\begin{document}

{\Huge\section*{Report}}

\tableofcontents
\vspace*{4em}



\phantomsection
\addcontentsline{toc}{section}{Problem 4, the former problem can be seen in the jog file in the same system.}
\subsection*{Problem 4, the former problem can be seen in the jog file in the same system.}



\phantomsection
\addcontentsline{toc}{section}{normalization algorithm}
\subsection*{normalization algorithm}



function [lambda] = eig(A, x, iteration\_cout)

\begin{lstlisting}
%	lambda = 0;
%	for k = 1:iteration_cout
%		v = A*x;
%		x = v/norm(v,2);
%		lambda = x'*A*x;
%	end
%
% end
\end{lstlisting}
\begin{lstlisting}[language={},xleftmargin=5pt,frame=none]

\end{lstlisting}
\begin{lstlisting}
A = [-261,209,-49;
	-530, 422, -98;
	-800,631,-144];
x = [1,0,0]';
lambda_1 = eig(A, x, 10)
lambda_2 = eig(A, x, 100)
lambda_3 = eig(A, x, 1000)
\end{lstlisting}
\begin{lstlisting}[language={},xleftmargin=5pt,frame=none]
lambda_1 =  10.000
lambda_2 = 10.0000
lambda_3 =  10.000

\end{lstlisting}


\end{document}
