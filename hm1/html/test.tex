

% This document was generated by the publish-function
% from GNU Octave 4.2.1



\documentclass[10pt]{article}
\usepackage{listings}
\usepackage{mathtools}
\usepackage{amssymb}
\usepackage{graphicx}
\usepackage{hyperref}
\usepackage{xcolor}
\usepackage{titlesec}
\usepackage[utf8]{inputenc}
\usepackage[T1]{fontenc}
\usepackage{lmodern}


\lstset{
language=Octave,
numbers=none,
frame=single,
tabsize=2,
showstringspaces=false,
breaklines=true}


\titleformat*{\section}{\Huge\bfseries}
\titleformat*{\subsection}{\large\bfseries}
\renewcommand{\contentsname}{\Large\bfseries Contents}
\setlength{\parindent}{0pt}

\begin{document}

{\Huge\section*{Homework 2}}

\tableofcontents
\vspace*{4em}



\phantomsection
\addcontentsline{toc}{section}{Quesion 1}
\subsection*{Quesion 1}



The random matrixs are A and B, matlab\textit{result is the result computed by matlab, and the your}result is the result computed by your naive method,
We use fro norm to calculate the error in computing matrix product

\begin{lstlisting}
fprintf("Raw matrix A:\n");
A = rand(5, 5);

fprintf("Raw matrix B:\n");
B = rand(5, 5);

fprintf("A*B by matlab\n");
matlab_result = A*B


fprintf("A*B by yourself\n");
your_result = calculate_matrix(A,B)

error_result = your_result.-matlab_result;

fprintf("Error result is:\n");
norm(error_result, 'fro')
\end{lstlisting}
\begin{lstlisting}[language={},xleftmargin=5pt,frame=none]
Raw matrix A:
Raw matrix B:
A*B by matlab
matlab_result =
   0.88264   0.67326   0.19422   0.81040   1.06861
   0.91942   1.03037   0.52540   0.88824   1.02150
   0.88454   1.13177   0.43811   1.18984   1.12484
   1.19768   0.73270   0.60775   1.01789   1.09111
   1.10078   1.02858   0.45187   1.00775   1.19491
A*B by yourself
your_result =
   0.88264   0.67326   0.19422   0.81040   1.06861
   0.91942   1.03037   0.52540   0.88824   1.02150
   0.88454   1.13177   0.43811   1.18984   1.12484
   1.19768   0.73270   0.60775   1.01789   1.09111
   1.10078   1.02858   0.45187   1.00775   1.19491
Error result is:
ans = 0

\end{lstlisting}


\phantomsection
\addcontentsline{toc}{section}{Question2 Test vector norm-1,2,inf}
\subsection*{Question2 Test vector norm-1,2,inf}

\begin{lstlisting}
random_vector = rand(100, 1);

% Calculate by the matlab

matlab_vector_1 = norm(random_vector, 1)
matlab_vector_2 =norm(random_vector, 2)
matlab_vector_inf = norm(random_vector, inf)


% calculate by yourself

x_vector_1 = sum(abs(random_vector))
x_vector_2 = sqrt(sum(random_vector.^2))
x_vector_inf = max(abs(random_vector))

error_1 = (x_vector_1-matlab_vector_1)/matlab_vector_1
error_2 = (x_vector_2-matlab_vector_2)/matlab_vector_2
error_inf = (x_vector_inf-matlab_vector_inf)/matlab_vector_inf
\end{lstlisting}
\begin{lstlisting}[language={},xleftmargin=5pt,frame=none]
matlab_vector_1 =  49.276
matlab_vector_2 =  5.6272
matlab_vector_inf =  0.96591
x_vector_1 =  49.276
x_vector_2 =  5.6272
x_vector_inf =  0.96591
error_1 = 0
error_2 = 0
error_inf = 0

\end{lstlisting}


\phantomsection
\addcontentsline{toc}{section}{Question2 Test vector norm-1,2,inf}
\subsection*{Question2 Test vector norm-1,2,inf}

\begin{lstlisting}
random_matrix = rand(10, 10);
n = size(random_matrix, 2);
m = size(random_matrix, 1);
matlab_matrix_1 = norm(random_matrix, 1)
matlab_matrix_2 = norm(random_matrix, 2)
matlab_matrix_inf = norm(random_matrix, inf)


x_matrix_1 = max(sum(abs(random_matrix)(:,1:n),1))

x_matrix_2 = sqrt(max(eig(random_matrix'*random_matrix)))


x_matrix_inf = max(sum(abs(random_matrix)(1:m,:),2))

(matlab_matrix_1-x_matrix_1)/matlab_matrix_1
(matlab_matrix_2-x_matrix_2)/matlab_matrix_2
(matlab_matrix_inf-x_matrix_inf)/matlab_matrix_inf
\end{lstlisting}
\begin{lstlisting}[language={},xleftmargin=5pt,frame=none]
matlab_matrix_1 =  6.6801
matlab_matrix_2 =  5.5423
matlab_matrix_inf =  6.6978
x_matrix_1 =  6.6801
x_matrix_2 =  5.5423
x_matrix_inf =  6.6978
ans = 0
ans =   -1.6026e-16
ans = 0

\end{lstlisting}


\end{document}
